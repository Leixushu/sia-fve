\documentclass[final,leqno,onefignum,onetabnum]{siamltex1213bueler}
% siamltex1213bueler.cls is a two or three line change of siamltex1213.cls to permit
% pdflatex to work and not spew warnings

\usepackage{amssymb,amsmath}

\usepackage{times}

% math macros
\newcommand\bv{\mathbf{v}}
\newcommand\bV{\mathbf{V}}
\newcommand\bn{\mathbf{n}}
\newcommand\bq{\mathbf{q}}
\newcommand\bQ{\mathbf{Q}}

\newcommand\CC{\mathbb{C}}
\newcommand{\DDt}[1]{\ensuremath{\frac{d #1}{d t}}}
\newcommand{\ddt}[1]{\ensuremath{\frac{\partial #1}{\partial t}}}
\newcommand{\ddx}[1]{\ensuremath{\frac{\partial #1}{\partial x}}}
\newcommand{\ddy}[1]{\ensuremath{\frac{\partial #1}{\partial y}}}
\newcommand{\ddxp}[1]{\ensuremath{\frac{\partial #1}{\partial x'}}}
\newcommand{\ddz}[1]{\ensuremath{\frac{\partial #1}{\partial z}}}
\newcommand{\ddxx}[1]{\ensuremath{\frac{\partial^2 #1}{\partial x^2}}}
\newcommand{\ddyy}[1]{\ensuremath{\frac{\partial^2 #1}{\partial y^2}}}
\newcommand{\ddxy}[1]{\ensuremath{\frac{\partial^2 #1}{\partial x \partial y}}}
\newcommand{\ddzz}[1]{\ensuremath{\frac{\partial^2 #1}{\partial z^2}}}
\newcommand{\Div}{\nabla\cdot}
\newcommand\eps{\epsilon}
\renewcommand{\grad}{\nabla}
\newcommand{\ihat}{\mathbf{i}}
\newcommand{\ip}[2]{\ensuremath{\left<#1,#2\right>}}
\newcommand{\jhat}{\mathbf{j}}
\newcommand{\khat}{\mathbf{k}}
\newcommand{\nhat}{\mathbf{n}}
\newcommand\lam{\lambda}
\newcommand\lap{\triangle}
\newcommand\Matlab{\textsc{Matlab}\xspace}
\newcommand\RR{\mathbb{R}}
\newcommand\vf{\varphi}

\title{Conservation in free-boundary layer models\thanks{This work was supported by NASA grant \# NNX13AM16G.}} 

\author{Ed Bueler\thanks{Dept.~of Mathematics and Statistics, and Geophysical Institute, University of Alaska Fairbanks (\texttt{elbueler@alaska.edu}).}}

\begin{document}
\maketitle
\slugger{siap}{xxxx}{xx}{x}{x--x}%slugger should be set to mms, siap, sicomp, sicon, sidma, sima, simax, sinum, siopt, sisc, or sirev

\begin{abstract}
FIXME
\end{abstract}

%\begin{keywords}\end{keywords}

%\begin{AMS}\end{AMS}


\pagestyle{myheadings}
\thispagestyle{plain}
\markboth{ED BUELER}{CONSERVATION IN FREE-BOUNDARY LAYER MODELS}

\section{Introduction}

Problems of this type appear for ice sheets \cite{JouvetBueler2012}, shallow water flows over marshes \cite{AlonsoSantillanaDawson}, [Dupuit-Forchheimer APPROX IN GROUNDWATER], subglacial hydrology \cite{AschwandenBuelerKhroulevBlatter,BuelervanPeltDRAFT,Schoofetal2012}, supraglacial runoff [CITE?], sea ice [CITE?], and tsunami run-up [CITE?], among other applications.

\section{Theory}

Let $\Omega \subset \RR^d$ be a bounded set with sufficiently-regular boundary.  The time-dependent model consists of a mass-conservation equation, a fixed-location boundary condition, and a constraint:
\begin{align}
u_t &= - \Div \bq + f(u,x,t), \label{eq:massconserve} \\
\bq \cdot \bn \Big|_{\partial\Omega} &= 0, \label{eq:fixedneumann} \\
u &\ge 0 \label{eq:constraint}
\end{align}
for $u(x,t)$ with $x\in \Omega$.  We suppose $\bq = \bq(\grad u, u, x, t)$.  Essentially because constraint \eqref{eq:constraint} comes from the meaning of $u$ as a layer thickness, equation \eqref{eq:massconserve} is sometimes called a ``St.~Venant'' equation [BECAUSE OF FIXME].

The time semi-discretized problem is a weakly-posed problem for the new values $u_n(x) \approx u(x,t_n)$, given the old values $u_{n-1}(x)$,
\begin{align}
\frac{u_n - u_{n-1}}{\Delta t} &= - \Div \bQ_n + F(u_n,x), \label{eq:semimassconserve} \\
\bQ_n \cdot \bn \Big|_{\partial\Omega} &= 0, \label{eq:semifixedneumann} \\
u_n &\ge 0 \label{eq:semiconstraint}
\end{align}
Here $F(u_n,x,t)$ is a very general function coming from the semi-discretization problem, and including various functions of the old values.  On the one hand, in the simplest implicit case, a backward Euler scheme, $F(u_n,x) = f(u_n,x,t_n)$ and $\bQ_n = \bq(\grad u_n,u,x,t_n)$.  In the case of a trapezoid rule, however,
\begin{align*}
F(u_n,x) &= \frac{1}{2} f(u_n,x,t_n) + \frac{1}{2} f(u_{n-1},x,t_{n-1}) - \frac{1}{2} \Div \bq(\grad u_{n-1},u_{n-1},x,t_{n-1}),
\end{align*}
and $\bQ_n = \frac{1}{2} \bq(\grad u_n,u_n,x,t_n)$.  Thus $F$ ``absorbs'' all the terms evaluated at time $t_{n-1}$.

Define
\begin{equation}
m(t) = \int_\Omega u(x,t)\,dx.
\end{equation}
Decompose $\Omega$ into three disjoint regions based on $u_n$ and $u_{n-1}$:
\begin{align*}
\Omega &= \Omega_n \cup \Omega_r \cup \Omega_0, \\
\Omega_n &= \{u_n(x)>0\}, \\
\Omega_r &= \{u_n(x)=0 \text{ but } u_{n-1}(x) > 0\}, \\
\Omega_0 &= \{u_n(x)=0 \text{ and } u_{n-1}(x) = 0\}.
\end{align*}
Here the subscript ``$r$'' stands for ``retreat''.  Note $\Omega_n$ has a boundary which decomposes
\begin{align*}
\partial\Omega_n &= \Gamma_f \cup \Gamma_0
\end{align*}
where along $\Gamma_f$ condition \eqref{eq:semifixedneumann} applies (``$f$'' stands for ``fixed'') and along $\Gamma_0$ we have $\bQ_n = 0$.

Then by \eqref{eq:semimassconserve} we have
\begin{align*}
m(t_{n}) - m(t_{n-1}) &= \int_{\Omega_n} (u_n - u_{n-1})\,dx - \int_{\Omega_r} u_{n-1}\,dx \\
   &= \Delta t \int_{\Omega_n} (- \Div \bQ_n + F(u_n,x)) \,dx - \int_{\Omega_r} u_{n-1}\,dx \\
   &= \Delta t \int_{\partial\Omega_n} \bQ_n \cdot \bn + \Delta t \int_{\Omega_n} F(u_n,x)\,dx - \int_{\Omega_r} u_{n-1}\,dx \\
   &= \Delta t \int_{\Omega_n} F(u_n,x)\,dx - \int_{\Omega_r} u_{n-1}\,dx
\end{align*}
because $\bQ_n=0$ along $\partial\Omega_n$.



\section{Conclusion}  FIXME


%         References
\bibliography{ice-bib}
\bibliographystyle{siam}

%\Appendix
%\section{FIXME}


\end{document}
