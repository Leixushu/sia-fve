\documentclass[final,leqno,onefignum,onetabnum]{siamltex1213bueler}
% siamltex1213bueler.cls is a two or three line change of siamltex1213.cls to permit
% pdflatex to work and not spew warnings

\usepackage{amssymb,amsmath}

\usepackage{times}

% math macros
\newcommand\bv{\mathbf{v}}
\newcommand\bV{\mathbf{V}}
\newcommand\bn{\mathbf{n}}
\newcommand\bq{\mathbf{q}}
\newcommand\bQ{\mathbf{Q}}

\newcommand\CC{\mathbb{C}}
\newcommand{\DDt}[1]{\ensuremath{\frac{d #1}{d t}}}
\newcommand{\ddt}[1]{\ensuremath{\frac{\partial #1}{\partial t}}}
\newcommand{\ddx}[1]{\ensuremath{\frac{\partial #1}{\partial x}}}
\newcommand{\ddy}[1]{\ensuremath{\frac{\partial #1}{\partial y}}}
\newcommand{\ddxp}[1]{\ensuremath{\frac{\partial #1}{\partial x'}}}
\newcommand{\ddz}[1]{\ensuremath{\frac{\partial #1}{\partial z}}}
\newcommand{\ddxx}[1]{\ensuremath{\frac{\partial^2 #1}{\partial x^2}}}
\newcommand{\ddyy}[1]{\ensuremath{\frac{\partial^2 #1}{\partial y^2}}}
\newcommand{\ddxy}[1]{\ensuremath{\frac{\partial^2 #1}{\partial x \partial y}}}
\newcommand{\ddzz}[1]{\ensuremath{\frac{\partial^2 #1}{\partial z^2}}}
\newcommand{\Div}{\nabla\cdot}
\newcommand\eps{\epsilon}
\renewcommand{\grad}{\nabla}
\newcommand{\ihat}{\mathbf{i}}
\newcommand{\ip}[2]{\ensuremath{\left<#1,#2\right>}}
\newcommand{\jhat}{\mathbf{j}}
\newcommand{\khat}{\mathbf{k}}
\newcommand{\nhat}{\mathbf{n}}
\newcommand\lam{\lambda}
\newcommand\lap{\triangle}
\newcommand\Matlab{\textsc{Matlab}\xspace}
\newcommand\RR{\mathbb{R}}
\newcommand\vf{\varphi}

\title{Conservation for fluid layers with free boundaries\thanks{Supported by NASA grant \# NNX13AM16G.}} 

\author{Ed Bueler\thanks{Dept.~of Mathematics and Statistics, and Geophysical Institute, University of Alaska Fairbanks (\texttt{elbueler@alaska.edu}).}}

\begin{document}
\maketitle
\slugger{siap}{xxxx}{xx}{x}{x--x}%slugger should be set to mms, siap, sicomp, sicon, sidma, sima, simax, sinum, siopt, sisc, or sirev

\begin{abstract}
FIXME
\end{abstract}

%\begin{keywords}\end{keywords}

%\begin{AMS}\end{AMS}


\pagestyle{myheadings}
\thispagestyle{plain}
\markboth{ED BUELER}{CONSERVATION FOR FLUID LAYERS WITH FREE-BOUNDARIES}

\section{Introduction}

Consider a fluid of one material type which moves over a solid substrate, or even a substrate which changes shape as the fluid moves.  Suppose further that fluid can be added or removed at the boundary of the fluid mass, either by a process like precipitation or aggregation (or even by phase change if we define the fluid region as of only one phase).  Through both flow and such boundary sources, the three-dimensional region occupied by the fluid changes in time, thus the problem is of moving-boundary type.  How can a numerical model of such free-boundary fluid motion, with non-material boundary surfaces, conserve the discretized mass of the fluid exactly?

Of course there is nothing special about ``mass'' in the above paragraph.  Can the discretized energy or momentum or mass of a fluid with moving boundary, and boundary source terms, be conserved exactly?

Problems of this type appear for ice sheets \cite{BLKCB,CDDSV,EgholmNielsen2010,JouvetBueler2012}, shallow water flows over marshes \cite{AlonsoSantillanaDawson}, [Dupuit-Forchheimer APPROX IN GROUNDWATER], laboratory flows \cite{Kondic,PeglerWorster2012,SayagWorster2013}, subglacial hydrology \cite{AschwandenBuelerKhroulevBlatter,BuelervanPeltDRAFT,Schoofetal2012}, supraglacial runoff (?), sea ice (?), and tsunami run-up (?), among other applications.  Within the context of ice sheet modeling, discrete mass conservation at free boundaries has a small literature \cite{Albrechtetal2011,JaroschSchoofAnslow2013} which suggests the difficulty of the problem.

Although the above literature relates to the problem we have stated, there is no theoretical guidance as to what degree such exact discrete conservation is possible.  The literature of global climate model addresses exact discrete conservation as a goal [FIXME: CITE], but always in a context without free boundary.  This paper starts a theory of exact discrete conservation for free-boundary fluid problems.

Our approach is to semi-discretize the problem in time and pose each time-step problem in weak variational form.  Each time step is a variational inequality because we look at the simplified case of fluid layers where the layer thickness is well-defined.  The positivity of thickness is the active constraint in the variational inequality form.

We identify the ``retreat area'' during the time step as fundamental.  By definition this area is the set where the fluid layer thickness was positive at the beginning of the time step, and, through flow and boundary-source terms, one-time-step weak problem solution gives thickness equal to zero.  That is, the fluid was lost from the retreat area at some time during the time-step.  Even for well-behaved source terms (e.g.~smooth in time and space) and short time steps, the retreat area can be of essentially arbitrary size.  For example, in a varying climate a large area of thin ice sheet or sea ice can melt, or a large area of a layer of surface water on ground can evaporate, in the duration of one time step no matter how short.  We can state our major assertion informally:
\begin{quote}
\emph{The retreat loss is not exactly balanced by a modeled source (either term or boundary integral) during the discrete time-step, so a conserving model must track it separately.}
\end{quote}

Climate models are ``multiphysics'' models which generally attempt to conserve masses of the phases of water separately, as these phases have different physical properties relevant to earth system dynamics.  For example, snow and ice have higher albedo than liquid ocean, and different densities as well.  In such climate multiphysics models with at least one fluid having a moving free boundary, auditable mass conservation does not generally occur.  We believe this is correctable once theoretical limits on discrete conservation are acknowledged, as here.  With the correction term identified here, for example, when a large area of ice sheet or sea-ice melts then the discrete-time model will both have exact conservation of the mass of water in all phases, but also it will be able to correctly report mass transfers between the phases for scientific purposes, as the different phases occupy changing two- and three-dimensional regions.


\section{Continuous-time and discrete-time strong formulations}  Let $\Omega \subset \RR^d$ be a bounded open region with sufficiently-regular boundary.  The time-dependent model we consider is usually stated in strong form as follows.  It combines a mass-conservation equation, a fixed-location flux (Neumann) boundary condition, and a constraint:
\begin{align}
u_t &= - \Div \bq + f(u,x,t) &&\text{in } \Omega, \text{ where } u > 0 \label{eq:massconserve} \\
\bq \cdot \bn &= g(x,t) &&\text{on } \partial\Omega, \text{ where } u > 0 \label{eq:fixedneumann} \\
u &\ge 0 &&\text{in } \bar\Omega. \label{eq:constraint}
\end{align}
The unknown is the \emph{layer thickness} $u(x,t)$, for $x\in \Omega$ and $t>0$.  The manner in which the flux $\bq$ depends on the unknown $u$ or its gradient $\grad u$ is undetermined, for now, but with more essential detail below.

Constraint \eqref{eq:constraint} comes from the meaning of $u$ as a thickness, which cannot be negative.  In the time-dependent support of $u$, namely $\{(x,t) \big| u(x,t) > 0\}$, we say that the layer exists, and that it is absent in the complement.  Finding the evolution of this support, and of its ``free'' boundary, is essential to problems of this type.

Of course, conservation equation \eqref{eq:massconserve} only applies in this strong form where the layer exists.  The same applies to the boundary condition \eqref{eq:fixedneumann}, which only makes sense in locations where $u>0$, at least in the generic case $g\ne 0$.  Well-posedness of the problem \eqref{eq:massconserve}--\eqref{eq:constraint} in fact requires a weak formulation, in which these caveats are replaced by a precise specification of admissible functions.  This is done below.

We actually work with the (time) semi-discretized problem.  Let $\{t_n\}$ be a sequence of increasing times.  The semi-discretized problem determines the new values $u_n(x) \approx u(x,t_n)$, given the old values $u_{n-1}(x)$.  Here is the strong form, corresponding to \eqref{eq:massconserve}--\eqref{eq:constraint} above:
\begin{align}
\frac{u_n - u_{n-1}}{\Delta t} &= - \Div \bQ_n + F_n &&\text{in } \Omega, \text{ where } u_n > 0 \label{eq:semimassconserve} \\
\bQ_n \cdot \bn &= G_n &&\text{on } \partial\Omega, \text{ where } u_n > 0 \label{eq:semifixedneumann} \\
u_n &\ge 0 &&\text{in } \bar\Omega \label{eq:semiconstraint}
\end{align}
Again, the problem \eqref{eq:semimassconserve}--\eqref{eq:semiconstraint} will be restated as a weak problem below, with attention to well-posedness.

In equations  \eqref{eq:semimassconserve}--\eqref{eq:semiconstraint}, the functions $F_n(u_n,x)$, $\bQ_n(\grad u_n,u_n,x)$, $G_n(x)$ come from the semi-discretization procedure, and including various combinations of the old and new values.  For example, in the simplest implicit case of a backward Euler scheme applied to \eqref{eq:massconserve}--\eqref{eq:constraint}, we have $F_n = f(u_n,x,t_n)$ and $\bQ_n = \bq(\grad u_n,u,x,t_n)$.  In the case of a trapezoid rule, however,
\begin{align*}
F_n &= \frac{1}{2} f(u_n,x,t_n) + \frac{1}{2} f(u_{n-1},x,t_{n-1}) - \frac{1}{2} \left(\Div \bq\right)(\grad u_{n-1},u_{n-1},x,t_{n-1}),
\end{align*}
and $\bQ_n = \frac{1}{2} \bq(\grad u_n,u_n,x,t_n)$.  Thus $F$ generally ``absorbs'' various terms evaluated at times $t_{n-1}$ and time $t_n$, but we will assume $F$ is zeroth-order in $u_n$.

FIXME: we can extend $G_n$ by zero to the whole of $\partial \Omega$ and not change anything.  On the flip side, as a modeler you are free to put $G_n$ zero anywhere on the boundary, but if it is nonzero then it must be that the solution $u_n$ is actually positive on that part of the boundary


\section{The retreat set, the retreat loss, and the unknown loss-time}  Decompose $\Omega$ into three disjoint regions based on $u_n$ and $u_{n-1}$:
\begin{align*}
\Omega_n &= \left\{x \in \Omega \,\big|\, u_n(x)>0\right\}, \\
\Omega_n^r &= \left\{x \in \Omega \,\big|\, u_n(x)=0 \text{ and } u_{n-1}(x) > 0\right\}, \\
\Omega_n^0 &= \left\{x \in \Omega \,\big|\, u_n(x)=0 \text{ and } u_{n-1}(x) = 0\right\},
\end{align*}
so that $\Omega = \Omega_n \cup \Omega_n^r \cup \Omega_n^0$.  Here the superscript ``$r$'' stands for ``retreat''.\footnote{At this point a symmetry has been broken.  We could have decomposed $\Omega= \Omega_n \cup \Omega_n^a \cup \Omega_n^0$ where $\Omega_n^a = \{u_n(x) > 0 \text{ and } u_{n-1}(x) = 0\}$ is the ``advance'' set.  As far as we can tell the resulting alternate theory offers no advantages \dots}  See Figure \ref{fig:domains}.

The boundary of the support $\Omega_n$ of $u_n$ decomposes into the part where a fixed (Neumann) condition applies, and a part which is the free boundary,
\begin{align*}
\partial\Omega_n &= \Gamma_n^N \cup \Gamma_n^0
\end{align*}
(superscript ``$N$'' stands for ``Neumann'').  Specifically, $\Gamma_n^0 = \Omega \cap \partial \Omega_n$ and $\Gamma_n^N = \partial \Omega \cap \partial \Omega_n$, and along $\Gamma_n^N$ the flux condition \eqref{eq:semifixedneumann} applies.  We will show [WILL WE?  PRESUMABLY ONLY IF $\bQ_n$ HAS $\grad u_n$] that along the free boundary $\Gamma_n^0$ we have both $u_n=0$ and $\bQ_n = 0$.

\begin{figure}[ht]
\begin{center}
\includegraphics[width=2.0in,keepaspectratio=true]{domains-fig}
\end{center}
\caption{We decompose $\Omega = \Omega_n \cup \Omega_n^r \cup \Omega_n^0$, where $\Omega_n$ the support of $u_n$, $\Omega_n^r$ is the retreat set, and $\Omega_n^0$ is the set on which both $u_{n-1}$ and $u_n$ are zero.  The boundary of $\Omega_n$ is decomposed into two pieces, $\partial\Omega_n = \Gamma_n^N \cup \Gamma_n^0$.}
\label{fig:domains}
\end{figure}

Now define
\begin{equation}
M_n = \int_\Omega u_n(x)\,dx, \label{eq:totalmassdefn}
\end{equation}
which we naturally call the \emph{(total) mass} at time $t_n$, and define
\begin{equation}
R_n = \int_{\Omega_n^r} u_{n-1}\,dx, \label{eq:retreatlossdefn}
\end{equation}
which we call the \emph{retreat loss} at time $t_n$.  The total mass and the retreat loss at time $t_n$ are related.  By \eqref{eq:semimassconserve}, \eqref{eq:totalmassdefn}, and \eqref{eq:retreatlossdefn} we have
\begin{align}
M_n - M_{n-1} &=  - \int_{\Omega_n^r} u_{n-1}\,dx + \int_{\Omega_n} (u_n - u_{n-1})\,dx \label{eq:massstep} \\
   &= - R_n + \Delta t \int_{\Omega_n} (- \Div \bQ_n + F_n) \,dx \notag \\
   &= - R_n + \Delta t \int_{\Gamma_n^N} G_n + \Delta t \int_{\Omega_n} F_n\,dx \notag
\end{align}
because $\bQ_n=0$ along $\Gamma_n^0$.

Given the continuous-time solution $u(x,t)$ starting with initial condition $u(x,t_{n-1}) = u_{n-1}(x)$, at points $x$ within $\Omega_n^r$ we could define the time at $u(x,t)$ first becomes zero, the \emph{loss-time function} which is well-defined on the retreat set $\Omega_n^r$:
\begin{equation}
\bar t(x) = \inf\left\{t \,\big|\, u(x,t)>0 \,\text{ and }\, t_{n-1} < t \le t_n\right\}.
\end{equation}
Observe that $\bar t(x)$ varies over $\Omega_n^r$.  Of course as $\Delta t \to 0$ then $\bar t(x) \to t_{n-1}$, and we might even expect that the area of $\Omega_n^r$ might decrease to zero, though this has not been proven.  But for numerical models, which necessarily have discrete time, the variation in $\bar t(x)$, over $\Omega_n^r$ during the time-step $t_{n-1} < t \le t_n$, is unknown.  If this uncertainty is not acknowledge then it might be a barrier to the conservation of discrete (at least, time-semi-discretized) mass.

As an operational statement about discrete-time models, we can rephrase our major assertion from the Introdcution as
\begin{quote}
\emph{The model must store a time series for $R_n$, in addition to the expected time series $\int_{\Gamma_n^N} G_n$ and $\int_{\Omega_n} F_n$, in order to provide auditable mass conservation.}
\end{quote}
In stating this assertion, we acknowledge that the retreat loss $R_n$ should vanish in the $\Delta t\to 0$ limit.


\section{Regarding the flux}

FIXME:  want property that for $X$ an open set,
\begin{equation}
u_n=0 \text{ on } X \quad \implies \quad \bQ_n=0 \text{ on } X  \label{eq:vanishingQn}
\end{equation}

FIXME:  want maximum principle property that
\begin{equation}
v + \alpha\, (\Div \bQ_n)(\grad v,v,x) > 0 \text{ on } X \quad \implies \quad v > 0 \text{ on } X  \label{eq:maxprincQn}
\end{equation}
for all $\alpha>0$ because the strong form \eqref{eq:semimassconserve} says $u_n + \Delta t\, \Div \bQ_n = (u_{n-1} + \Delta t\, F_n)$ and we want to conclude the contrapositive of \eqref{eq:maxprincQn} in $\Omega_n^0$ and $\Omega_n^r$ where $u_n=0$:
\begin{equation}
F_n \le 0  \text{ on } \Omega_n^0  \label{eq:inequalityonzero}
\end{equation}
\begin{equation}
u_{n-1} + \Delta t\, F_n \le 0  \text{ on } \Omega_n^r  \label{eq:inequalityonretreat}
\end{equation}

FIXME: if $v \in W^{1,p}(\Omega)$ then, for $1/p + 1/q = 1$,
    $$\bQ_n(\grad v,v,x) \in L^q(\Omega)$$ 

\section{Weak formulation of a time-step}  The strong form \eqref{eq:semimassconserve}--\eqref{eq:semiconstraint} is generally understood to be inadequate as a description of the solutions $u_n(x)$, because the free boundary is not organically included in the problem statement, and also because the space of admissible solutions is not specified.  Here we both specify the appropriate function spaces and propose a weak form, a variational inequality \cite{Friedman,KinderlehrerStampacchia} for \eqref{eq:semimassconserve}--\eqref{eq:semiconstraint}.  This weak form can be proven to be well-posed in some cases.  In more cases the weak form can be shown to imply the strong form where the solution exists (``interior condition'').

We start by arguing informally for why the weak form, a non-obvious variational inequality, should hold.\footnote{Though the weak form is mathematically more fundamental, nonetheless history and human frailty have made the strong form more prominent in applications, especially in the climate-modeling literature where our conservation concerns are most relevant.}  Our argument uses facts which are intuitively true in the subsets of the decomposition $\Omega = \Omega_n \cup \Omega_n^r \cup \Omega_n^0$.

Suppose $v\ge 0$ is sufficiently smooth on $\Omega$.  Using the decomposition and the divergence theorem (Green's theorem),
\begin{align*}
-\int_{\Omega} \bQ_n \cdot \grad(v-u_n) &= -\int_{\Omega_n} \bQ_n \cdot \grad(v-u_n) - \int_{\Omega_n^r \cup \Omega_n^0} \bQ_n \cdot \grad(v-u_n) \\
  &= \int_{\Omega_n} (\Div \bQ_n) (v-u_n) - \int_{\Omega_n} \Div \left(\bQ_n (v-u_n)\right) \\
  &\qquad\quad + \int_{\Omega_n^r \cup \Omega_n^0} (\Div \bQ_n) (v-u_n) - \int_{\Omega_n^r \cup \Omega_n^0} \Div \left(\bQ_n (v-u_n)\right) \\
  &= \int_{\Omega_n} (\Div \bQ_n) (v-u_n) - \int_{\Gamma_n^N} G_n (v-u_n) \\
  &\qquad\quad + \int_{\Omega_n^r \cup \Omega_n^0} (\Div \bQ_n) (v-u_n)
\end{align*}
because
       $$\int_{\Gamma_n^0} (\bQ_n \cdot \bn) (v-u_n) = 0,$$
and using \eqref{eq:semifixedneumann}.  By \eqref{eq:semimassconserve} where the layer exists, namely in $\Omega_n$, we have
\begin{align}
-\int_{\Omega} \bQ_n \cdot \grad(v-u_n) &= \int_{\Omega_n} \left(F_n - \frac{u_n - u_{n-1}}{\Delta t}\right) (v-u_n) - \int_{\partial \Omega} G_n (v-u_n) \label{eq:equalitybeforeVI} \\
  &\qquad\quad + \int_{\Omega_n^r \cup \Omega_n^0} (\Div \bQ_n) (v-u_n). \notag
\end{align}
Here we have used the fact that $G_n$ extends by zero to the whole of $\partial \Omega$.

However, by \eqref{eq:vanishingQn} on $\Omega_n^0$,
    $$\int_{\Omega_n^0} (\Div \bQ_n) (v-u_n) = \int_{\Omega_n^0} (0) (v-u_n) \ge \int_{\Omega_n^0} \left(F_n - \frac{u_n - u_{n-1}}{\Delta t}\right) (v-u_n)$$
because in fact $u_n=u_{n-1}=0$ and $v-u_n = v \ge 0$ on $\Omega_n^0$, and because \eqref{eq:inequalityonzero} says that $F_n \le 0$ on $\Omega_n^0$.  Almost the same, by \eqref{eq:vanishingQn} on $\Omega_n^r$,
    $$\int_{\Omega_n^r} (\Div \bQ_n) (v-u_n) = \int_{\Omega_n^r} (0) (v-u_n) \ge \int_{\Omega_n^r} \left(F_n - \frac{u_n - u_{n-1}}{\Delta t}\right) (v-u_n)$$
because in fact $u_n=0$ and $v-u_n = v \ge 0$ on $\Omega_n^0$, and because \eqref{eq:inequalityonretreat} says that $F_n - (u_n - u_{n-1})/\Delta t = F_n + u_{n-1}/\Delta t \le 0$ on $\Omega_n^0$.  Thus if we return to \eqref{eq:equalitybeforeVI} we have
\begin{align}
-\int_{\Omega} \bQ_n \cdot \grad(v-u_n) &\ge \int_{\Omega_n} \left(F_n - \frac{u_n - u_{n-1}}{\Delta t}\right) (v-u_n) - \int_{\partial \Omega} G_n (v-u_n) \label{eq:essentiallyVI} \\
  &\qquad\quad + \int_{\Omega_n^r \cup \Omega_n^0} \left(F_n - \frac{u_n - u_{n-1}}{\Delta t}\right) (v-u_n). \notag
\end{align}

We want to be able to pose our problem (weakly) so that the decomposition $\Omega = \Omega_n \cup \Omega_n^r \cup \Omega_n^0$ is not needed to pose the problem.  But \eqref{eq:essentiallyVI} can be written without that decomposition!:
\begin{equation}
-\int_{\Omega} \bQ_n \cdot \grad(v-u_n) \ge \int_{\Omega} \left(F_n - \frac{u_n - u_{n-1}}{\Delta t}\right) (v-u_n) - \int_{\partial \Omega} G_n (v-u_n) \label{eq:morallytheVI}
\end{equation}
This is our variational inequality weak form, in which the free boundary does not appear in posing the problem.

\begin{definition}  Fix $p>1$.  Let
    $$\mathcal{K} = \left\{v \in W^{1,p}(\Omega) \,\big|\, v(x) \ge 0 \text{ for all } x \in \Omega\right\}.$$
\end{definition}

\begin{definition}  We say $u_n \in \mathcal{K}$ \emph{solves the one-time-step (weak) problem} if 
\begin{align}
\int_{\Omega} u_n (v-u_n) - \Delta t\, \bQ_n \cdot \grad(v-u_n) + &\Delta t \int_{\partial \Omega} G_n (v-u_n) \ge \label{eq:theVI} \\
  &\qquad\qquad \int_{\Omega} \left(u_{n-1} + \Delta t F_n\right) (v-u_n) \notag
\end{align}
\end{definition}
for all $v \in \mathcal{K}$.



\section{Conclusion}  FIXME


%         References
\bibliography{ice-bib}
\bibliographystyle{siam}

%\Appendix
%\section{FIXME}


\end{document}
