\documentclass[11pt,reqno]{amsart}
%prepared in AMSLaTeX, under LaTeX2e
\addtolength{\oddsidemargin}{-.65in}
\addtolength{\evensidemargin}{-.65in}
\addtolength{\topmargin}{-.3in}
\addtolength{\textwidth}{1.5in}
\addtolength{\textheight}{.6in}

\renewcommand{\baselinestretch}{1.1}

\usepackage{verbatim} % for "comment" environment

\usepackage[pdftex, colorlinks=true, plainpages=false, linkcolor=blue, citecolor=red, urlcolor=blue]{hyperref}

\newtheorem*{thm}{Theorem}
\newtheorem*{defn}{Definition}
\newtheorem*{example}{Example}
\newtheorem*{problem}{Problem}
\newtheorem*{remark}{Remark}

\newcommand{\mtt}{\texttt}
\usepackage{alltt,xspace}
\usepackage[normalem]{ulem}
\newcommand{\mfile}[1]
{\medskip\begin{quote}\scriptsize \begin{alltt}\input{#1.m}\end{alltt} \normalsize\end{quote}\medskip}

\usepackage[final]{graphicx}
\newcommand{\mfigure}[1]{\includegraphics[height=2.5in,
width=3.5in]{#1.eps}}
\newcommand{\regfigure}[2]{\includegraphics[height=#2in,
keepaspectratio=true]{#1.eps}}
\newcommand{\widefigure}[3]{\includegraphics[height=#2in,
width=#3in]{#1.eps}}

% macros
\usepackage{amssymb}

\usepackage[T1, OT1]{fontenc}
\renewcommand{\dh}{\fontencoding{T1}\selectfont{\symbol{240}}}

\newcommand{\bod}{B\"o\dh varsson\xspace}
\newcommand{\bods}{B\"o\dh varsson's}
\newcommand{\citebod}{B\"o\dh varsson (1955)\xspace}
\newcommand{\citepbod}{(B\"o\dh varsson, 1955)\xspace}

\newcommand{\bA}{\mathbf{A}}
\newcommand{\bB}{\mathbf{B}}
\newcommand{\bE}{\mathbf{E}}
\newcommand{\bF}{\mathbf{F}}
\newcommand{\bJ}{\mathbf{J}}
\newcommand{\br}{\mathbf{r}}
\newcommand{\bx}{\mathbf{x}}
\newcommand{\hbi}{\mathbf{\hat i}}
\newcommand{\hbj}{\mathbf{\hat j}}
\newcommand{\hbk}{\mathbf{\hat k}}
\newcommand{\hbn}{\mathbf{\hat n}}
\newcommand{\hbr}{\mathbf{\hat r}}
\newcommand{\hbt}{\mathbf{\hat t}}
\newcommand{\hbx}{\mathbf{\hat x}}
\newcommand{\hby}{\mathbf{\hat y}}
\newcommand{\hbz}{\mathbf{\hat z}}
\newcommand{\hbphi}{\mathbf{\hat \phi}}
\newcommand{\hbtheta}{\mathbf{\hat \theta}}
\newcommand{\complex}{\mathbb{C}}
\newcommand{\ppr}[1]{\frac{\partial #1}{\partial r}}
\newcommand{\ppt}[1]{\frac{\partial #1}{\partial t}}
\newcommand{\ppx}[1]{\frac{\partial #1}{\partial x}}
\newcommand{\ppy}[1]{\frac{\partial #1}{\partial y}}
\newcommand{\ppz}[1]{\frac{\partial #1}{\partial z}}
\newcommand{\pptheta}[1]{\frac{\partial #1}{\partial \theta}}
\newcommand{\ppphi}[1]{\frac{\partial #1}{\partial \phi}}
\newcommand{\pp}[2]{\frac{\partial #1}{\partial #2}}
\newcommand{\ppp}[2]{\frac{\partial^2 #1}{\partial^2 #2}}
\newcommand{\pppp}[3]{\frac{\partial^2 #1}{\partial #2 \partial #3}}
\newcommand{\Div}{\ensuremath{\nabla\cdot}}
\newcommand{\Curl}{\ensuremath{\nabla\times}}
\newcommand{\curl}[3]{\ensuremath{\begin{vmatrix} \hbi & \hbj & \hbk \\ \partial_x & \partial_y & \partial_z \\ #1 & #2 & #3 \end{vmatrix}}}
\newcommand{\cross}[6]{\ensuremath{\begin{vmatrix} \hbi & \hbj & \hbk \\ #1 & #2 & #3 \\ #4 & #5 & #6 \end{vmatrix}}}
\newcommand{\eps}{\epsilon}
\newcommand{\grad}{\nabla}
\newcommand{\image}{\operatorname{im}}
\newcommand{\integers}{\mathbb{Z}}
\newcommand{\ip}[2]{\ensuremath{\left<#1,#2\right>}}
\newcommand{\lam}{\lambda}
\newcommand{\lap}{\triangle}
\newcommand{\Matlab}{\textsc{Matlab}\xspace}
\newcommand{\exers}[1]{\bigskip\noindent\textbf{Exercises} #1}
\newcommand{\fexer}[2]{\bigskip\noindent\textbf{Lesson #1, \##2}\quad }
\newcommand{\prob}[1]{\bigskip\noindent\textbf{#1} }
\newcommand{\pts}[1]{(\emph{#1 pts}) }
\newcommand{\epart}[1]{\medskip\noindent\textbf{(#1)}\quad }
\newcommand{\ppart}[1]{\,\textbf{(#1)}\quad }
\newcommand{\note}[1]{[\scriptsize #1 \normalsize]}
\newcommand{\MatIN}[1]{\mtt{>> #1}}
\newcommand{\onull}{\operatorname{null}}
\newcommand{\rank}{\operatorname{rank}}
\newcommand{\range}{\operatorname{range}}
\renewcommand{\P}{\mathcal{P}}
\newcommand{\real}{\mathbb{R}}
\newcommand{\trace}{\operatorname{tr}}
\renewcommand{\Re}{\operatorname{Re}}
\renewcommand{\Im}{\operatorname{Im}}
\newcommand{\Arg}{\operatorname{Arg}}

\newcommand{\comm}[2]{\item \emph{#1}:\, #2}

\renewcommand{\ln}[2]{\comm{line #1}{#2}}
\newcommand{\lnpage}[3]{\comm{line #1 \underline{on page #2}}{#3}}
\newcommand{\lns}[2]{\comm{lines #1}{#2}}
\newcommand{\lnspage}[3]{\comm{lines #1 \underline{on page #2}}{#3}}
\newcommand{\fg}[2]{\comm{Figure #1}{#2}}
\newcommand{\eqn}[2]{\comm{equation #1}{#2}}

\newcommand{\reply}[2]{
\medskip\medskip
\item  \begin{quote}
\emph{#1}
\end{quote}

\medskip
\noindent #2}


\title[Author's replies to reviews of \emph{Stable finite volume element schemes \dots}]{Author's replies to reviews of \\ \emph{Stable finite volume element schemes} \\ \emph{for the shallow ice approximation}}

\author{Ed Bueler}

\date{\today}

\begin{document}
\maketitle

\thispagestyle{empty}



I want to thank both reviewers and the Scientific Editor for encouraging and helpful comments on, and questions about, the manuscript.  I hope I have addressed them all satisfactorily.


\subsection*{Reviewer \#1}  \begin{itemize}
\reply{This paper revisits the classical Mahaffy finite difference scheme to solve the steady
non-sliding shallow ice approximation (SIA) combined with the mass conservation equation.  The aim is two-fold: First, this allows us to get new insights [into] Mahaffy's
scheme, which can be reinterpreted by finite element and finite volume. Second, this
re-interpretation allows us to improve substantially the original Mahaffy scheme by
choosing for better quadrature (however, without impacting the stencil) and by up-winding the most ``advective'' part, which involves the gradient of the bed.\\
This is a very well written and very instructive paper, which presents an original
method to solve one of the most useful ice flow models. Although the paper is restricted
to a simple case (steady, non-sliding, isothermal SIA), the ideas presented here can be
easily generalized to more physical situations. Thus, I have no doubt it will interest the
community of glacier and ice sheet modellers. Beyond its originality, I found the paper
enjoyable reading, and especially didactic. This is even more remarkable considering
the technical and mathematical nature of the paper.}
{This summary is accurate, and appreciated!}

\reply{I have reported below [a] few points of discussion, which could lead to some possible
improvements. However, none of those points are critical, and I believe that the paper
can be published as it is.}
{This recommendation is also appreciated.}

\reply{General comments: \medskip \\
The scheme $M^\star$ presents two major improvements to original Mahaffy's scheme.
Fig.~6 shows strong evidences of improvements in term of convergence under
refinement due to improved quadratures. In contrast, the evidences for the up-
winding sound weaker since restricted to the step-bedrock case. I'm wondering
if an additional experiment with non flat bed but smooth (even with a constant
bed slope), could further highlight the advantages of treating the part involving
$\grad b$ by an upwinding strategy through splitting (6).  In this experiment, one could
compare the method performances of both methods with and without splitting
(6) while increasing the bed slope. In addition, it would interesting to check what
value of $\lambda$ is optimal in another case (I was left wondering how much the amount
of upwinding $\lambda$ is case-dependent, is $\lambda = 1/4$ a robust value?).}
{The evidence for good performance in the bedrock-step case \emph{is} weaker, compared to the flat bed case, for two reasons: (1) There is only Jarosch et al.'s exact solution to use as a precise tool.  (2) The Jarosch exact solution has extraordinarily low regularity (e.g.~not even continuity!) so even convergence in $L^\infty$ and $L^1$ norms is not expected.  I believe the paper explains this adequately; see Reviewer Jarosch's comment below which confirms the situation.

It is a good question, which could be explored, of how robust is $\lambda=1/4$ is.  Of course one could construct further experiments, with smooth non-flat beds as suggested by Reviewer Jouvet.  However, explaining these experiments would require other numerical approximations, I think, because only benchmark solutions would be available, not exact solutions.  (For example, the non-flat, smooth-bed solution used by Jouvet \& Bueler (2012) requires an admittedly-highly-accurate ODE solution.)  Explaining the experiments would also add length to the paper.  In this regard, in designing this paper I prioritized having scalability, demonstrated on the Greenland examples, over further verification.

On the other hand, let me point out that I \emph{do} show that smoother nonflat beds give better convergence of the Newton iteration than rougher.  Namely, in the ``BM1'' Greenland runs at the higher resolutions (1250m, 1000m, 625m) the bed is quite smooth on the small scale because, as explained in the text, the gridded bed comes from bilinear interpolation of the original 5 km data.  The consequence shown in Figure 9 is clear: Newton-iteration convergence is much better on the BM1-based smoother beds than on the rough ``MCB'' beds at comparable resolutions (e.g.~1200m,900m,600m).}

\reply{Continuing with this aspect, I would find interesting to see how the upwinding improves the conditioning of the Jacobian matrix, as said line 256.}
{This is an excellent point, but I have not made a careful study of this issue.  A discussion of this is beyond the scope of a \emph{J.~Glaciol.} article I believe.  

It is my hope that a good SIAM paper can/should be written about this problem looking at two things that are mentioned here referee comments and in my replies: (1) attempt to analyze and/or improve the $M^\star$ FVE method by looking at streamline diffusion applied to the un-upwinded method, (2) examination of conditioning of the Jacobian it relates to bed and surface roughness.  I suspect these go together.}

\reply{I understand the ``step-by-step'' motivation of the continuation method proposed
in the paper.  However, I find it hard to justify since only the last solution ($\epsilon_{12}$)
has a physical meaning. \dots}
{There is a literature on continuation as a Newton globalization strategy.  I am using continuation in exactly the way it is used across a large area of scientific computing.  See Kelley (2003), cited in my paper, but also see the modern review by Knoll \& Keyes (2004).\footnote{D.~Knoll and D.~Keyes (2004). \emph{Jacobian-free Newton-Krylov methods: a survey of approaches and applications.} J.~Computational Phys., 193(2), 357--397.}

The idea behind continuation is indeed that only the final ``parameter $=0$'' result is desired, but of course the ball of quadratic convergence for the Newton iteration may be small.  In time-stepping, as mentioned by the Reviewer below, this is not an issue.  The difficulty in the steady-state case is that a simple initial iterate construction, like in formula (39) here, does not bring one close to the ball of quadratic convergence.  A continuation strategy is a globalization; it gets to a point inside the ball of quadratic convergence by a fixed amount of work, such as my strategy of a fixed number of continuation levels.

I assert this strategy is preferred to a only-linearly-convergent fixed-point iteration.  By contrast, the Picard strategy suggested by the Reviewer (below) would either require either abandoning quadratic convergence as a goal, or implementing two matrix-evaluation routines (one for Picard and one for the Jacobian) and using the Picard iteration as a globalization strategy.  Note that the continuation strategy only requires the implementation of one parameterized Newton solver, with residual- and Jacobian-evaluation routines, which are both used in the continuation stages ($\epsilon>0$) and the final stage ($\epsilon=0$).}

\reply{\dots  In contrast the time-stepping is meaningful.  Maybe,
it would be beneficial to argue why the conclusions based on solutions obtained
with the continuation method remain valid when computing physical solutions
(like time-stepping solutions).}
{This is a bit confused, I think.

The problem here is that the bed-roughness-caused-low-regularity plus free-boundary-caused-low-regularity, of the (unknown) exact continuum solution or of the continuum equations which define that exact solution, makes the ball of quadratic convergence for the Newton solver very small.  Time-stepping is a poor way to approach steady state, that is, it is a poor globalization strategy for trying to get an iterate inside that ball (Knoll \& Keyes, 2004).  Almost any other Newton globalization strategy is better; I used continuation.

Maybe the best way to address the Reviewer's concern is to point out that once I ``give up'' and turn to time-stepping, the continuation strategy is not really important.  In particular, I say in lines 446--448, about the high-resolution Greenland simulation on the roughest beds, that ``These implicit time-steps are chosen sufficiently-short so that the continuation scheme fully converges (i.e.~to the $\epsilon_{12}=0$ level), \dots'', and it is later stated that these are 0.1 year steps on this very fine 900m grid.  One could shorten the time steps further and skip the continuation entirely, but I recommend using continuation to lengthen the (low-regularity-caused) time-step restriction; it is more efficient.}

\reply{Alternatively to the continuation method, one could first use Picard iterations,
and switch to Newton iterations after getting sufficiently close to the solution.
It seems that this has become a common practise in recent models. Have you
considered this approach?  \dots}
{Yes, this approach has been considered.  The fact that a lot of people do Picard is not compelling here because I find the finite-diffence evaluation of Jacobians quite effective, so I drafted my PETSc-using code by just writing a residual-evaluation routine.  This was easier than implementing a Picard strategy, and it did most of the job, including giving quadratic convergence on all the lower-resolution cases.  As mentioned in the paper, I also implemented the full analytical Jacobian, and it gives small benefits (e.g.~1.5 times faster and more robustness with respect to bed roughness at higher resolutions).

Now simply repeating my comment above, the Picard strategy would either require either abandoning quadratic convergence as a goal, or implementing two matrix-evaluation routines (one for Picard and one for the Jacobian) and using the Picard iteration as a globalization strategy.  Note that the continuation strategy only requires the implementation of one parameterized Newton solver, with residual- and Jacobian-evaluation routines, which are both used in the continuation stages ($\epsilon>0$) and the final stage ($\epsilon=0$).}

\reply{\dots In [1], we have proposed a simple continuation scheme
between Picard and Newton, which allows to switch from to Picard to the Newton
by varying a simple parameter $\gamma$ from 0 to 1, considering that Newton’s scheme
can be seen as Picard’s scheme, however with additional terms, which involve
higher derivatives. For instance, in the case of the simple $p$-Laplace problem with
zero Dirichlet boundary condition, one can solve the non-linear problem \medskip \\
\phantom{foobar} [\emph{variational equation for $p$-laplacian Poisson equation}] \medskip \\
by solving instead the sequence of linearized problems: \medskip \\
\phantom{foobar} [\emph{a linearization equation}] \medskip \\
which [is] Picard's method when $\gamma=0$, Newton’s method when $\gamma=1$,
and an hybrid combination Picard-Newton when $\gamma \in (0,1)$.  Maybe, varying $\gamma$
continuously from 0 to 1 could be an interesting alternative to the continuation
method presented in the paper?}
{This is indeed an interesting alternative, and I am aware of it, but I did not try it.

My primary concern here is not about efficiency but instead robustness with respect to bed roughness.  In this suggested alternative, are the Picard-like end cases ($\gamma\approx 0$) less sensitive to the bed gradient, as are the larger-$\epsilon$ cases in my continuation strategy?  Does $0<\gamma<1$ generate a larger region of superlinear convergence when the problem is not a minimization (e.g.~the $p$-Laplacian shown here) but instead like the non-flat-bed SIA?

Alternatively, one could only implement the Picard matrix but then use it as a preconditioner for the Newton-Krylov iteration.  This strategy is advocated by Knoll \& Keyes (2004), and PETSc facilitates it, but for my application the full analytical Jacobian was not excessive work.  This may, however, be a good strategy when the steady-state free-boundary problem for the mass-conservation equation is using a membrane-stress-resolving stress balance, and one already has an implemented separate Picard method for the stress balance alone.}

\reply{l.~42 Gauss-Seidel}
{Fixed.}

\reply{eq.~(15) I think that $m_{j,k}$ was not introduced at that point.}
{It is now defined.}

\reply{eq.~(22) $y_{k} + 0$ and $y_k − 0$?}
{I tried to get away with it, but that did not work.  I now use standard limit notation in (22).}

\reply{l.~299,301,306,307 Lips\textbf{c}hitz}
{Right!  No more ``shit''; sorry about that.}

\reply{Fig. 6 Can the little convergence circles be drawn further visible? I can hardly see them
when printed in black and white.}
{Good point.  The circles have been made much more visible.}

\reply{l.~394,395 ``However, maximum errors ... norm convergence,'' this is probably a hard sentence for most of JOG readers, but I don't know how to better explain.}
{I agree with the comment, that many \emph{J.~Glaciol.} readers will not be used to this kind of sentence.  However, it transmits an important idea, so I tried to rewrite it for clarity.  It now says:
\begin{quote}
However, maximum errors are not expected to decay.  (This is because merely interpolating a discontinuous function like the exact solution with piecewise-linear functions generates large errors in the maximum norm.)
\end{quote}}

\reply{l.~396 ``suggest'' $\implies$ ``ensure''?}
{Good idea.  Done.}

\reply{Fig.~8, caption I would recall in the caption the meaning of $M^\star$, $M_{no}^\star$ and $M_{full}^\star$, but also that the thick line is the step-bedrock.}
{Agreed.  In fact, the labels ``$M_{no}^\star$'' and ``$M_{full}^\star$'' used in Figure 8, Table 1, and the text, are now gone.  I use ``$\lambda=0$'' and ``$\lambda=1$'' for these concepts, with an appropriate mention of equation (31) to explain.  And the fact that the thick line is the bedrock is now stated in the caption.}

\reply{l.~437 to 450 Just an idea: what about summarizing the steps into a scheme/table instead of
a paragraph? I think that would be clearer.}
{This suggestion makes sense because the narrative description in this paragraph is not elegant.

However, it is sufficiently heteromorphous and heterogeneous so that turning it into a table may not even be possible.  It could be written as a ``recipe'' with enumerated steps, but I would not want to do that because the reader might assume its details are strongly recommended in some sense.  This awkward paragraph documents what was done, e.g.~for reproducability and so as to make errors diagnosable, but this paragraph is \emph{not} stating a recommended recipe.}
\end{itemize}


\subsection*{Reviewer \#2}  \begin{itemize}
\reply{The manuscript at hand presents a significantly improved numerical scheme for shallow ice models. It is based on a re-interpretation of the classical existing numerics and an novel, elegant extension into ``finite volume elements'' that allows for an efficient solving routine of the discrete equations that are constrained for non-negative ice thicknesses. The new scheme proofs itself to be stable by being fully implicit, performs outstandingly well in comparison to exact solution (with and without bed topography) and is able to solve large scale, real world examples (e.g.~Greenland) efficiently. Especially noteworthy is the inclusion of the ice thickness constraint ($H\ge 0$), that is tackled by a parallel Newton scheme, as it allows for prognostic modeling that generates realistic ice margins without any mass violations.  A ``feature'' that almost none of the currently available ice flow models include.}
{This summary is accurate and appreciated.

I do want to be careful with the reviewer's phrase ``ice margins without any mass violations,'' as I do not claim this in the paper.  The schemes I describe are indeed all locally mass-conserving in the sense that the numerical flux out of one ice-filled cell is exactly equal to the flux into the ice-filled cell which shares a face with it.  However, in steady state, such a condition cannot be true between an ice-filled cell and an adjacent ice-free cell, something most easily seen by considering a one-dimensional FV-type steady state.  (I am writing a separate paper about such issues---e.g.~inevitable conservation errors in discretized free-boundary models---and their consequences.)

Again, my main point regarding the reviewer's comment is that I do not claim that the margins are violation-free when it comes to mass conservation.  I \emph{do} claim to give a good numerical approximation of what I believe is the correct (well-posed) steady SIA \emph{continuum} problem, which includes the positive thickness constraint, and which has a free-boundary.}

\reply{The manuscript is very well written and presents the science in a concise and detailed manner, which is easily comprehensible to the wider audience.  Thus I have only a few minor comments to make, which I list below:
\medskip \\
Line \# 8: As this is a single author paper, I was surprised to read ``we'' throughout the manuscript. This might just be me, being a non-native English speaking person, but it strikes me as rather awkward.}
{I have apparently become comfortable with the ``we'' style from frequently reading it in mathematics textbooks.  For a discussion and explanation of the style, including a recommendation that ``we'' is the default style when a personal pronoun is needed, see\begin{quote}
\href{http://academia.stackexchange.com/questions/2945/choice-of-personal-pronoun-in-single-author-papers}{\texttt{academia.stackexchange.com/questions/2945/\\ \phantom{foobar} choice-of-personal-pronoun-in-single-author-papers}}
\end{quote}
This discussion includes this core advice from Paul Halmos, an authority on writing mathematics in English: \begin{quote} \medskip
There is nothing wrong with the editorial ``we'', but if you like it, do not misuse it.  Let ``we'' mean ``the author and the reader''\dots  \, Thus, it is fine to say ``Using Lemma 2 we can generalize Theorem 1,'' or ``Lemma 3 gives us a technique for proving Theorem 4.''  It is not good to say ``Our work on this result was done in 1969'' (unless the voice is that of two authors, or more, speaking in unison), \dots \medskip \\
\indent The use of ``I,'' and especially its overuse, sometimes has a repellent effect, as arrogance or ex-cathedra preaching, and, for that reason, I like to avoid it whenever possible. \dots \medskip
\end{quote}
Based on this and similar advice I have received over time, I would like to stick with the ``we'' style.  I recognize that the \emph{Journal of Glaciology} is not a mathematics textbook.  Readers of the \emph{Journal} rarely see a single-author paper (!) and may have come to assume ``we'' means the multiple authors.

However, I have reviewed the paper for the unnecessary use of personal pronouns and have removed several instances of ``we'', especially in the abstract where it may be most distracting.}

\reply{Line \# 88: Maybe replace ``Glen-law'' with ``Glen's flow law'' or something like that to not confuse the ``freshman'' generation to ice flow modeling.}
{Done.}

\reply{Line \# 90: ``ice softness A'' is fine, as is ``power n'', but again a quick reference to Glen's law might help the newcomers.}
{Done.}

\reply{Equations (3),(4),(5),(6) are all stating different ways to calculate the flux q. It might be worthwhile to denote those with different indices or ``add-on'' symbols like $q'$, $q'$, etc. and only leave the $q$ in (6). This might help to avoid confusion in the uncareful reader who might wonder which $q$ is meant in Line \# 112. A reference to eq (6) might do the trick as well.}
{I want to emphasize that formulas (3), (4), (5), and (6) all compute the \emph{same} flux.  Thus I \emph{do not} want to use different symbols for the ``$\mathbf{q}$'' they compute; it is the \emph{same} flux!  The $\mathbf{q}$ in line 112 is the same flux as everywhere else.  Whether the flux is diffusive is \emph{not} about which choice you have made for its appearance, e.g.~among the four formulas, but rather it is intrinsic to it.  The only question is whether the flux is always anti-parallel to the gradient of the conserved quantity; in that case it is diffusive.

I have amplified this point in the text.  I have also combined formula (7) into (6) so the four flux forms are described in more similar ways.}

\reply{Line \# 137. Again the unmindful reader might wonder, why equation (3) is solved as there is a flux definition in the novel eq.~(6), not realizing that the diffusivity ``D'' is still required and defined in (3). Maybe change to: ``solve coupled equations (6), (7), and (9) with D from (3).''}
{I have changed the text to indicate that (3) only supplies $D$.}

\reply{Equation (12): It might be worthwhile to state that this is the ``classical'' M2 scheme as classified by Hindmarsh and Payne 1996. This might help the wider audience to realize that a specific, yet very common scheme for calculating the ice thickness at the flux boundary is chosen here.}
{Yes, good point.  I have cited H\&P 1996.}

\reply{Equation (24): If I am not completely mistaken here (due to the lack of coffee) the $\alpha_\triangle$ should have the power ($n-1)$ as in eq.~(11) and not $(n-1)/2$.}
{Fixed.}

\reply{Line \# 260: $W^*_x$ should be $W^x_*$ as in eq.~(30) and the rest of the text.}
{Fixed.}

\reply{Line \# 391--400 (including Fig.~8 and Table 1): It is nice to see that the exact solution from Jarosch et al.~2013 turns out to be useful.  The author details here why thickness errors (max and average) are not converging with grid refinement and that is the same reason why Jarosch et al.~2013 did not report such error norms, which in hindsight might have been a useful measure.  Based on the weaker volume error measure, it is again very nice to see that the M* scheme outperforms the overly complex MUSCL-superbee scheme of Jarosch et al.~2013.}
{I definitely appreciate this explanation!

I think higher-resolution flux schemes, such as MUSCL, based on the ``hyperbolic'' flux factorizations (4), (5), or (6), are a promising idea.  I do not know \emph{why} the compact stencil version here performs better.  The exact solution from Jarosch et al.~(2013) was the first precise tool with which to examine the issue; I wish there were more such tools!

I have the idea that a FEM-style streamline diffusion explanation may be illuminating, \emph{a la} the discussion of that topic in the book Elman et al.~(2005).  This is outside the scope of my paper, however.}

\reply{Figure 9: There should be a label on the y-axis, $\epsilon_i$ maybe?}
{Yes, that's reasonable.  Done.}

\reply{Line \# 434: Maybe add a reference that explains the ``RS'' and ``SS'' methods for the less numerically inclined readers.}
{Good point.  It is the same Benson \& Munson (2006) paper cited elsewhere, but indeed this is a good place for a citation.}

\reply{Line \# 442: There is no $\epsilon_{12}$ displayed in Figure 9.  Did you mean $\epsilon_{11}$?}
{I have clarified in the text that a ``good'' level might mean $\epsilon_i<10^{-3}$, so $\epsilon_{10}$--$\epsilon_{12}$ is the appropriate range.  I no longer mention Figure 9 here; that was not the point.}

\reply{I hope the newly presented scheme becomes soon a feature in PISM to facilitate the wider use of an inequality constraint (ice thickness), mass conserving, stable and elegant scheme like this one.}
{What we really want with PISM is a version of this work, e.g.~a steady-state and/or fully-implicit free-boundary scheme, suitable for the combination of the SIA+SSA hybrid stress balance and the mass-conservation equation.  Or suitable for the hydrostatic/Blatter-Pattyn equations.  We are not there yet, unfortunately.}

\reply{\dots  I also highly appreciate (as should the glaciological community as a whole) that the author instantly publishes his code on which the presented research is based on along with the manuscript on Github for open access, reproducibility and transparency!}
{Thanks!}
\end{itemize}

\end{document}